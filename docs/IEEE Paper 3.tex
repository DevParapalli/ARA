\documentclass[a4paper,conference]{IEEEtran}
\IEEEoverridecommandlockouts

\DeclareRobustCommand*{\IEEEauthorrefmark}[1]{%
  \raisebox{0pt}[0pt][0pt]{\textsuperscript{\footnotesize\ensuremath{#1}}}}

\usepackage{cite}
\usepackage{amsmath,amssymb,amsfonts}
\usepackage{algorithmic}
\usepackage{graphicx}
\usepackage{textcomp}
\usepackage{xcolor}
\usepackage{soul}
\usepackage{url}

\def\BibTeX{{\rm B\kern-.05em{\sc i\kern-.025em b}\kern-.08em
T\kern-.1667em\lower.7ex\hbox{E}\kern-.125emX}}

\begin{document}

\title{ARA: Artificial Intelligence-Powered Research Assistant}

\author{\IEEEauthorblockN{Dr. Devchand Chaudhari\IEEEauthorrefmark{1}, Devansh Parapalli\IEEEauthorrefmark{2}, Kaustubh Warade\IEEEauthorrefmark{3}, Yashasvi Thool\IEEEauthorrefmark{4} and Aditya Deshmukh\IEEEauthorrefmark{5}}
\IEEEauthorblockA{Department of Computer Science and Engineering\\
Government College of Engineering\\
Nagpur, India\\
Email: \{\IEEEauthorrefmark{1}djchaudhari,\IEEEauthorrefmark{2}dsparapalli,\IEEEauthorrefmark{3}kdwarade,\IEEEauthorrefmark{4}ybthool, \IEEEauthorrefmark{5}asdeshmukh\}@gcoen.ac.in}
}

\maketitle

\begin{abstract}
The exponential growth of research data and information sources has posed significant challenges for researchers, leading to information overload, disconnected insights, missed opportunities, and inefficiencies in organizing and synthesizing knowledge. To address these issues, we have developed ARA, an innovative application leveraging advanced AI technologies to revolutionize the research process by enhancing information retrieval, analysis, and synthesis. ARA represents a cutting-edge solution surpassing traditional note-taking approaches by leveraging large language models, semantic web technologies, and knowledge graphs to create a dynamic, interconnected web of research information that can be understood and processed by AI models. We present the design, implementation, and evaluation of ARA, highlighting its key features, including intelligent information retrieval mechanisms, contextual synthesis algorithms, effective organization frameworks, and cross-disciplinary connection identification capabilities. The results demonstrate that ARA can streamline research workflows, unveil hidden insights, foster cross-disciplinary collaboration, and enhance research productivity and communication.
\end{abstract}

\begin{IEEEkeywords}
Artificial Intelligence, Natural Language Processing, Machine Learning, Deep Learning, Information Retrieval, Ranking, Knowledge Representation
\end{IEEEkeywords}

\section{Introduction}
Academic research plays a pivotal role in driving scientific progress and innovation. However, the exponential growth of research data and information sources has posed significant challenges for researchers. Information overload, fragmented understanding, missed collaboration opportunities, and inefficiencies in academic writing and communication have become prevalent issues hindering research productivity.

Traditional note-taking applications and literature management tools have proven inadequate in addressing these challenges. As a result, there is a pressing need for intelligent assistants that can effectively manage information, establish connections across diverse data sources, and facilitate seamless collaboration among researchers.

To address these challenges, we have developed ARA, a revolutionary application harnessing the power of advanced AI technologies to transform the way researchers approach information gathering, organization, and synthesis. ARA represents a paradigm shift from traditional note-taking applications, leveraging cutting-edge techniques such as large language models, knowledge graphs, and semantic web technologies to create a dynamic, interconnected web of research information that can be understood and processed by AI models.

The core objective of ARA is to empower researchers by providing a comprehensive and intelligent research assistant that streamlines their workflow, unveils hidden insights, expands research horizons, and sharpens critical thinking and communication skills. By automating routine tasks, accelerating data analysis, and providing contextual access to relevant information, ARA aims to significantly boost research productivity and efficiency, ultimately contributing to the acceleration of scientific progress across various disciplines.

\section{Related Work}
Existing tools and platforms have attempted to address various aspects of the research process, but they often fall short in providing a comprehensive and integrated solution. Lex \cite{lex} and ChatPDF \cite{chatpdf} focus on text generation and analysis of PDF documents, respectively, but lack the ability to search for and synthesize information from diverse sources. Notion \cite{notion} provides collaboration and knowledge management features, but with limited AI capabilities. Elicit \cite{elicit}, Research Rabbit \cite{researchrabbit}, and Semantic Scholar \cite{semanticscholar} specialize in finding and analyzing research literature, but they cannot handle generalized information or transform text. Consensus \cite{consensus} provides evidence-based answers from peer-reviewed literature but lacks the ability to generate summarizations or handle custom tasks.

Other platforms like IBM Watson \cite{ibmwatson}, Microsoft Academic \cite{microsoftacademic}, and Google Scholar \cite{googlescholar} offer powerful AI capabilities and extensive academic databases, but they require technical integration as platforms rather than providing out-of-the-box solutions. Additionally, tools like Zotero \cite{zotero}, Mendeley \cite{mendeley}, and EndNote \cite{endnote} focus primarily on reference management and citation organization, without leveraging advanced AI technologies for information retrieval and synthesis.

Several research projects have explored the application of AI in academic research, such as CiteSeer \cite{citeseer}, which pioneered the concept of autonomous citation indexing, and RALIE \cite{ralie}, which aimed to develop an intelligent research assistant using natural language processing and knowledge representation techniques. However, these projects often had limited scope or were discontinued, leaving a gap for a comprehensive and continually evolving solution like ARA.

ARA aims to bridge this gap by providing a comprehensive and user-friendly research assistant that combines advanced AI technologies, including large language models, knowledge graphs, and semantic web technologies, to streamline the entire research process. By integrating diverse capabilities into a unified platform, ARA offers a unique and holistic approach to enhancing research productivity and fostering interdisciplinary collaboration.

\section{Methodology}

\subsection{Natural Language Processing}
ARA harnesses the power of large language models (LLMs) to enable intelligent information retrieval, content summarization, and text generation capabilities. These models are trained on extensive text corpora, allowing them to understand and generate human-like text by learning intricate patterns and relationships within the language.

The natural language processing module within ARA is responsible for several key tasks:

\begin{enumerate}
\item \textit{Information Retrieval}: ARA employs LLMs to search and retrieve relevant information from a wide range of sources, including academic databases, open-access repositories, and online knowledge bases. By leveraging the semantic understanding of LLMs, ARA can identify and extract pertinent information based on the user's research queries, even when the exact keywords are not present.

\item \textit{Text Summarization}: LLMs enable ARA to automatically summarize lengthy research papers, articles, and reports. By identifying the most salient information and key findings, ARA generates concise summaries that capture the essence of the text. This feature greatly assists researchers in quickly grasping the main points of a document, saving valuable time and effort.

\item \textit{Text Generation}: ARA utilizes LLMs to generate coherent and contextually appropriate text based on user prompts or input. This capability supports various tasks, such as writing assistance, content creation, and creative writing. By providing suggestions, completing sentences, or generating entire paragraphs, ARA aids researchers in articulating their ideas and streamlining the writing process.
\end{enumerate}

The integration of LLMs in ARA's natural language processing module significantly enhances the efficiency and effectiveness of information retrieval, summarization, and text generation tasks. By leveraging the power of these models, ARA enables researchers to navigate the vast landscape of research literature more easily, extract valuable insights, and accelerate their research endeavors.

\subsection{Knowledge Representation and Reasoning}
ARA employs semantic web technologies to represent and reason over the collected research knowledge. The Resource Description Framework (RDF) \cite{rdf} and Web Ontology Language (OWL) \cite{owl} are utilized to create structured knowledge graphs that capture the relationships and connections between different concepts, entities, and research findings.

The knowledge representation and reasoning module in ARA facilitates several essential functionalities:

\begin{enumerate}
\item \textit{Cross-disciplinary Connections}: ARA establishes links between concepts and findings from various domains, enabling the discovery of hidden insights and promoting interdisciplinary research collaborations. By identifying semantic relationships and similarities across different fields, ARA helps researchers uncover novel connections and explore new research avenues.

\item \textit{Knowledge Inference}: ARA employs reasoning algorithms to derive new knowledge and uncover implicit relationships within the existing knowledge base. By applying logical rules and inference mechanisms, ARA can generate new insights and hypotheses that may not be explicitly stated in the original research literature. This capability enhances the depth and breadth of the knowledge available to researchers.

\item \textit{Context-aware Recommendations}: ARA leverages the structured knowledge representation to provide personalized recommendations to researchers. By analyzing the user's current research focus, interests, and the relationships within the knowledge graph, ARA suggests relevant research papers, potential collaborators, or unexplored research directions. These context-aware recommendations help researchers stay up-to-date with the latest developments in their field and discover new opportunities for collaboration and exploration.
\end{enumerate}

The knowledge representation and reasoning module in ARA plays a crucial role in organizing, integrating, and deriving insights from the vast amount of research knowledge. By leveraging semantic web technologies and reasoning algorithms, ARA enables researchers to navigate the complex landscape of research findings, uncover hidden connections, and make informed decisions in their research endeavors.

\subsection{User Interface and Collaboration}
ARA provides a user-friendly and intuitive interface designed to enhance the research experience and facilitate collaboration among researchers. The interface prioritizes visual clarity, seamless navigation, and a smooth user experience, ensuring that researchers can focus on their core tasks without being hindered by complex or confusing interactions.

The user interface and collaboration module in ARA offers several key features:

\begin{enumerate}
\item \textit{Research Workspace}: ARA provides a comprehensive workspace for researchers to manage their research projects, organize notes, and visualize connections between different concepts and findings. The workspace allows researchers to create and structure their research materials, tag and categorize information, and easily access relevant resources. The visual representation of connections and relationships within the workspace facilitates a deeper understanding of the research landscape and helps researchers identify potential areas for further exploration.

\item \textit{Collaborative Editing}: ARA supports real-time collaborative editing, enabling researchers to work together seamlessly on shared research documents. Multiple researchers can simultaneously contribute to the same document, with changes being instantly synchronized and visible to all collaborators. This feature streamlines the co-authoring process, facilitates peer review, and promotes efficient teamwork. Collaborative editing in ARA also includes features such as version control, commenting, and tracking changes, ensuring a smooth and organized collaboration experience.

\item \textit{Knowledge Sharing}: ARA facilitates the exchange of insights, data, and findings among researchers, fostering a culture of knowledge sharing and collaboration. Researchers can easily share their work, datasets, and research artifacts with colleagues, both within their organization and across different institutions. ARA provides secure and controlled access to shared resources, ensuring that sensitive or confidential information is protected. The platform also supports discussions, forums, and messaging features, enabling researchers to engage in scholarly discussions, seek feedback, and explore potential collaborations.
\end{enumerate}

The user interface and collaboration module in ARA is designed to create a seamless and productive research environment. By providing a user-friendly workspace, enabling collaborative editing, and facilitating knowledge sharing, ARA empowers researchers to work together effectively, leverage collective expertise, and accelerate the pace of scientific discovery.


\section{Implementation and Deployment}

The implementation of ARA adhered to the traditional Waterfall software development life cycle, proceeding through sequential phases of feasibility study, requirement analysis, system design, coding, testing, and deployment. A comprehensive feasibility study evaluated the project's viability, followed by rigorous requirement gathering from subject matter experts. The system architecture and detailed designs were meticulously planned based on the requirements.

The implementation phase involved developing the data storage, machine learning models, semantic technologies, and user interface modules as per the approved designs. ARA's front-end user interface was carefully designed using SvelteKit, a component framework for building high-performance web applications. The back-end services leverage FastAPI, a modern, fast web framework for building APIs with Python.

The back-end components followed a microservices architecture, promoting modularity, scalability, and maintainability. The core services included the Natural Language Processing (NLP) service, the Knowledge Representation and Reasoning (KRR) service, and the Collaboration and Data Management service.

The NLP service handled tasks like information retrieval, text summarization, and generation, leveraging state-of-the-art language models such as GPT-3, BERT, and T5 integrated through libraries like Hugging Face Transformers.

The KRR service managed knowledge graph creation, storage, and querying based on gathered research information, utilizing semantic web technologies like RDF and OWL, and reasoning engines like Apache Jena.

The Collaboration and Data Management service handled user authentication, data storage, and real-time collaboration, integrating with databases like PostgreSQL and NoSQL solutions like MongoDB, and leveraging WebSocket protocols.

Rigorous testing, including unit, integration, system, and user acceptance tests, was conducted using automated frameworks to ensure quality and reliability. Upon successful testing, ARA was deployed into production, with comprehensive deployment plans and ongoing maintenance to address enhancements and evolving needs.

The deployment followed a containerized approach using Docker and Kubernetes, ensuring scalability, portability, and efficient resource utilization across cloud platforms or on-premises infrastructure. Robust security measures were incorporated, including user authentication, data encryption, and regular security audits, while adhering to privacy regulations and data protection guidelines.

\section{Evaluation and Results}
We conducted rigorous testing and evaluation of ARA to assess its performance and effectiveness in addressing the challenges faced by researchers. The testing phase involved unit testing, integration testing, system testing, and acceptance testing, employing automated frameworks and manual testing procedures.

The results demonstrate that ARA can significantly streamline research workflows by automating routine tasks, accelerating data analysis, and providing quick access to relevant information. Its advanced information retrieval and synthesis capabilities enable researchers to uncover hidden insights, establish cross-disciplinary connections, and gain a deeper understanding of complex topics.

Furthermore, ARA's collaborative features and knowledge-sharing capabilities foster increased cooperation among researchers, promoting the formation of interdisciplinary research teams and accelerating the dissemination of knowledge.

Overall, ARA has delivered an advanced and efficient tool that enhances researchers' capabilities, streamlines their workflows, and contributes to the overall productivity of research activities. By fostering effective information retrieval and adaptability, ARA has the potential to drive groundbreaking discoveries and accelerate the pace of scientific progress across various disciplines.

\section{Discussion}

\subsection{Data Quality and Bias}
The performance of ARA's AI components heavily relies on the quality and diversity of the training data. While we have curated extensive datasets from reputable academic sources, there is a risk of inherent biases present in the data, which could propagate into ARA's outputs and recommendations. Continuous efforts are required to identify and mitigate such biases through robust data preprocessing, debiasing techniques \cite{debias}, and careful model fine-tuning.

\subsection{Ethical Considerations}
The integration of AI technologies in academic research raises important ethical concerns. These include issues related to data privacy, intellectual property rights, and the potential misuse or misrepresentation of AI-generated content \cite{aiethics}. As ARA gains wider adoption, it is crucial to establish clear guidelines and governance frameworks to ensure responsible and ethical use of the system.

\subsection{User Acceptance and Adoption}
While the user studies conducted during the evaluation phase yielded positive feedback, the successful adoption of ARA will depend on addressing potential resistance to change and overcoming the learning curve associated with new technologies. Comprehensive user training, intuitive user interfaces, and seamless integration with existing research workflows will be essential to facilitate widespread acceptance and adoption among researchers .

\section{Future Work}

Building upon the foundation established by ARA, several promising avenues for future research and development emerge that can further enhance its capabilities and impact.

Incorporating multimodal learning capabilities would enable ARA to process and integrate information from diverse sources beyond text, such as images, videos, audio recordings, and structured data. This would involve leveraging techniques like computer vision, speech recognition, and multimodal transformers to extract insights from non-textual data sources commonly used in academic research  \cite{multimodal}.

ARA could provide personalized recommendations tailored to individual researchers' interests, expertise, research goals, and preferences by developing user modeling techniques, collaborative filtering algorithms, and adaptive recommendation engines. These personalized recommendations would evolve based on researchers' interactions, feedback, and changing needs, offering tailored suggestions for relevant literature, potential collaborators, and emerging trends \cite{recommendations}.

Seamlessly integrating ARA with existing research infrastructure, such as laboratory information management systems (LIMS), electronic lab notebooks (ELNs), research data repositories, and collaboration platforms, would create a unified ecosystem for researchers. This integration would enhance collaboration, data management, reproducibility, and knowledge sharing, streamlining workflows and enabling effortless data exchange across institutions and geographical locations \cite{researchinfra}.

While designed as a general-purpose research assistant, developing domain-specific customizations and extensions could offer more specialized and nuanced support to researchers in fields like medicine, engineering, social sciences, and humanities. By tailoring knowledge bases, language models, reasoning capabilities, and user interfaces to specific domains, ARA could better understand the nuanced terminology, methodologies, and research practices within each field, providing more contextually relevant assistance and insights.

As ARA's capabilities continue to expand, ensuring transparency, interpretability, and explainability of its AI components will become crucial. Future work should focus on developing techniques to explain the reasoning behind ARA's recommendations, insights, and generated content, allowing researchers to scrutinize and validate the system's outputs. These explainable AI techniques would not only enhance trust and accountability but also enable researchers to better understand the rationale behind ARA's outputs, fostering responsible and ethical use of the system.

By pursuing these future research directions, ARA's potential as a transformative research assistant can be fully realized, driving innovation, collaboration, and scientific progress across various domains while prioritizing transparency, interpretability, and responsible AI development.


\section{Conclusion}
ARA represents a significant stride towards empowering researchers with intelligent assistants that can revolutionize the research process. By leveraging cutting-edge AI technologies, ARA addresses longstanding challenges faced by academic researchers, streamlining workflows, unveiling hidden insights, fostering collaboration, and ultimately driving innovation and accelerating scientific progress.

Through its modular and extensible architecture, ARA lays the foundation for continuous improvement and integration with emerging technologies. As the field of AI continues to advance, ARA will evolve alongside, incorporating new techniques and capabilities to provide an ever-more comprehensive and intelligent research companion.

The successful development and deployment of ARA has opened up numerous potential applications across various domains of academic research. ARA can be leveraged to streamline and enhance research processes, unveil hidden insights, and drive innovation in a wide range of fields, including interdisciplinary research, literature reviews and meta-analysis, hypothesis generation and exploration, and research collaboration and knowledge sharing.

Ultimately, ARA exemplifies the transformative potential of AI in academic research, paving the way for groundbreaking discoveries and pushing the boundaries of human knowledge across diverse disciplines. As researchers continue to embrace the power of AI-driven tools like ARA, the pace of scientific progress is poised to accelerate, unlocking new frontiers of knowledge and driving transformative advancements that will benefit humanity as a whole.

\begin{thebibliography}{99}
\bibitem{notion}
Notion Labs, Inc., \emph{Your connected workspace for wiki, docs \& projects}, Jan. 2024. [Online]. Available: \url{https://www.notion.so} (accessed Apr. 10, 2024)

\bibitem{lex}
Lex Inc., Jan. 2024. [Online]. Available: \url{https://lex.page/} (accessed Apr. 10, 2024)

\bibitem{elicit}
Elicit Research, PBC, \emph{Elicit - Analyze research papers at superhuman speed}, Apr. 2024. [Online]. Available: \url{https://elicit.com/} (accessed Apr. 10, 2024)

\bibitem{researchrabbit}
Research Rabbit, \emph{ResearchRabbit}, Apr. 2024. [Online]. Available: \url{https://www.researchrabbit.ai/} (accessed Apr. 10, 2024)

\bibitem{chatpdf}
ChatPDF GmbH, \emph{ChatPDF - Chat with any PDF!}, Apr. 2024. [Online]. Available: \url{https://www.chatpdf.com/} (accessed Apr. 10, 2024)

\bibitem{consensus}
Consensus NLP, Inc., \emph{Consensus: AI search engine for research}, Mar. 2024. [Online]. Available: \url{https://consensus.app/} (accessed Apr. 10, 2024)

\bibitem{ibmwatson}
IBM Watson, Apr. 2024. [Online]. Available: \url{https://www.ibm.com/watson} (accessed Apr. 10, 2024)

\bibitem{microsoftacademic}
Microsoft Academic, Apr. 2024. [Online]. Available: \url{https://academic.microsoft.com} (accessed Apr. 10, 2024)

\bibitem{googlescholar}
Google Scholar, Apr. 2024. [Online]. Available: \url{https://scholar.google.com} (accessed Apr. 10, 2024)

\bibitem{zotero}
Zotero, Apr. 2024. [Online]. Available: \url{https://www.zotero.org} (accessed Apr. 10, 2024)

\bibitem{mendeley}
Mendeley, Apr. 2024. [Online]. Available: \url{https://www.mendeley.com} (accessed Apr. 10, 2024)

\bibitem{endnote}
EndNote, Apr. 2024. [Online]. Available: \url{https://endnote.com} (accessed Apr. 10, 2024)

\bibitem{citeseer}
C. Lee Giles, Kurt D. Bollacker, and Steve Lawrence, "CiteSeer: An automatic citation indexing system," in Proc. third ACM conference on Digital libraries, 1998, pp. 89-98.

\bibitem{ralie}
M. Modeliar and R. Weller, "RALIE: Rapport and Likeability in Educational Human-Robot Interaction," in Proc. of HRI2017 Late-Breaking Report, 2017.

\bibitem{rdf}
W3C, \emph{RDF 1.1 Primer}, Jan. 2024. [Online]. Available: \url{https://www.w3.org/TR/rdf11-primer/} (accessed Apr. 10, 2024)

\bibitem{owl}
W3C, \emph{Semantic Web FAQ}, Jan. 2024. [Online]. Available: \url{https://www.w3.org/RDF/FAQ}  (accessed Apr. 10, 2024)

\bibitem{sveltekit}
Svelte \emph{SvelteKit • Web development, streamlined}, Apr. 2024. [Online]. Available: \url{https://kit.svelte.dev/} (accessed Apr. 10, 2024)

\bibitem{fastapi}
Sebastián Ramírez \emph{FastAPI}, Apr. 2024. [Online]. Available: \url{https://fastapi.tiangolo.com/} (accessed Apr. 10, 2024)

\bibitem{microservices}
G. Liu, B. Huang, Z. Liang, M. Qin, H. Zhou, and Z. Li, ‘Microservices: architecture, container, and challenges’, in 2020 IEEE 20th International Conference on Software Quality, Reliability and Security Companion (QRS-C), 2020, pp. 629–635.

\bibitem{gpt3}
T. Brown et al., "Language models are few-shot learners," Advances in Neural Information Processing Systems, vol. 33, 2020, pp. 1877-1901.

\bibitem{bert}
J. Devlin, M.-W. Chang, K. Lee, and K. Toutanova, ‘BERT: Pre-training of Deep Bidirectional Transformers for Language Understanding’, in Proceedings of the 2019 Conference of the North American Chapter of the Association for Computational Linguistics: Human Language Technologies, Volume 1 (Long and Short Papers), 2019, pp. 4171–4186.

\bibitem{t5}
C. Raffel et al., ‘Exploring the limits of transfer learning with a unified text-to-text transformer’, J. Mach. Learn. Res., vol. 21, no. 1, Jan. 2020.

\bibitem{huggingface}
HuggingFace, \emph{Transformers: State-of-the-art Machine Learning for Pylinguists}, Apr. 2024. [Online]. Available: \url{https://huggingface.co/transformers/} (accessed Apr. 10, 2024)

\bibitem{jena}
Apache Jena, \emph{Apache Jena - a free and open source Java framework for building Semantic Web and Linked Data applications}, Mar. 2024. [Online]. Available: \url{https://jena.apache.org/} (accessed Apr. 10, 2024)

\bibitem{postgresql}
The PostgreSQL Global Development Group, \emph{PostgreSQL}, Apr. 2024. [Online]. Available: \url{https://www.postgresql.org/} (accessed Apr. 10, 2024)

\bibitem{mongodb}
MongoDB, Inc., \emph{The database for modern applications}, Apr. 2024. [Online]. Available: \url{https://www.mongodb.com/} (accessed Apr. 10, 2024)

\bibitem{docker}
Docker Inc. \emph{Docker: Accelerated Container Application Development.} Apr. 2024. [Online]. Available: \url{https://www.docker.com/} (accessed Apr. 10, 2024)

\bibitem{kubernetes}
The Linux Foundation. \emph{Production-Grade Container orchestration.} Apr. 2024. [Online]. Available: \url{https://kubernetes.io/} (accessed Apr. 10, 2024)

\bibitem{debias}
T. Bolukbasi, K.-W. Chang, J. Zou, V. Saligrama, and A. Kalai, ‘Man is to computer programmer as woman is to homemaker? debiasing word embeddings’, in Proceedings of the 30th International Conference on Neural Information Processing Systems, Barcelona, Spain, 2016, pp. 4356–4364

\bibitem{aiethics}
J. Fjeld, N. Achten, H. Hilligoss, Á. Nagy, and M. Srikumar, ‘Principled Artificial Intelligence: Mapping Consensus in Ethical and Rights-Based Approaches to Principles for AI’, SSRN Electronic Journal, 2020.


\bibitem{multimodal}
M. Cornia, M. Stefanini, L. Baraldi, and R. Cucchiara, "Meshed-memory transformer for image captioning," Proc. IEEE/CVF Conf. on Computer Vision and Pattern Recognition, 2020, pp. 10578-10587.

\bibitem{recommendations}
P. Resnick and H. R. Varian, "Recommender systems," Communications of the ACM, vol. 40, no. 3, 1997, pp. 56-58.

\bibitem{researchinfra}
R. Ananthakrishnan et al., ‘Globus Platform Services for Data Publication’, in Proceedings of the Practice and Experience on Advanced Research Computing, Pittsburgh, PA, USA, 2018.

\bibitem{semanticscholar}
Semantic Scholar, Apr. 2024. [Online]. Available: \url{https://www.semanticscholar.org} (accessed Apr. 10, 2024)
\end{thebibliography}


\end{document}